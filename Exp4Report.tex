\documentclass{article}
\usepackage{amsmath, amssymb, amsthm}
\usepackage{inputenc}
\usepackage{geometry}
\usepackage{graphicx}
\usepackage{setspace}
\usepackage[version=4]{mhchem}
\geometry{legalpaper, portrait, margin=1in}
\title{Empirical Formula of Hydrated Copper (II) Sulfate: 
The Accuracy and Precision of a Classic General Chemistry Experiment\\Experiment 14}
\author{Dylan VanAllen}
\date{Thursday \\ 3/11/2021}
\doublespacing
\begin{document}
    \begin{singlespace}
    \maketitle
    \end{singlespace}
    \section*{Objective}
    The goal of this experiment is to determine an empirical formula of Hydrated Copper (II) Sulfate. The empirical formula 
    is the chemical formula that shows only relative numbers of atoms of each type in a molecule. By doing an experiment that determines 
    the ratio of moles of one element to another we can empirically determine the formula. 
    \section*{Hypothesis}
    The experiment will confirm the empirical formula to be \ce{CuSO4(g) * 5H2O(s)}. This will be done by comparing the experimental mole 
    ratio of \ce{SO4} to \ce{Cu} and the mole ratio of \ce{H2O} to \ce{CuSO4}. Mass is easier to measure and deal with, and is conserved. That 
    is why mass is measured rather than moles. Since we are starting with 3 grams of \ce{CuSO4(g) * 5H2O(s)}, the prediction is that it will 
    contain: \(0.1 \pm 0.01\ g\) \ce{H2O}, \(0.07 \pm 0.003\ g\) \ce{Cu}, and \(0.1 \pm 0.01\ g\) \ce{SO4}. The molar masses will be used to find the 
    moles of each substance. Using reaction \((1)\), one can compare mole ratios of each product to expected values.
    \section*{Variables}
    The independent variable is the mass of the \ce{CuSO4(g) * 5H2O(s)}. The independent variables are the mole ratios of each product. 
    \section*{Experimental Outline}
    Heat \(3 \ g\) of \ce{CuSO4(g) * 5H2O(s)} in a test tube and wait until the reaction precipitates. Then determine the mass of the precipitate \ce(Cu), 
    the mass of the \ce{H2O}, and the mass of the \ce{SO4}. Using their molar masses, determine the mole ratios and compare them to expected values. The 
    expected mole ratio of \ce{H2O} to \ce{CuSO4} is 5, and the expected mole ratio of \ce{So4} to \ce{Cu} is 1. 
    \section*{Chemical Hazards and Waste}
    Wear goggles, gloves, and a lab coat. Be careful with bunsen burners, and turn off the gas when done. Dispose of any acid and do not get the 
    \ce{CuSO4(g) * 5H2O(s)} in eyes or on skin. 
    \section*{Key Math Equations}
    \begin{align}
    \ce{C(s)} + \ce{O2(g) &-> CO2(g)} \\
    \ce{C3H8(g)} + \ce{5O2(g) &-> 3CO2(g)} + \ce{4H2O(g)}\\
    \ce{CuSO4(g) * 5H2O(s) &-> CuSO4(s)} + \ce{5H2O(g)} \\
    \ce{CuSO4(aq)} + \ce{Mg(s) &-> MgSO4(aq)} + \ce{Cu(s)} \\
    \ce{Mg(s)} + \ce{2HCl(aq) &-> MgCl2(aq)} + \ce{H2(g)} 
    \end{align}
    
    \section*{Data}
    \includegraphics[width=15cm]{images/exp4data.jpg}
    \section*{Conclusion}
    The empirical formula for \ce{CuSO4(g) * 5H2O(s)} was confirmed from the experiment with very high accuracy (indicated by low percent errors). 
    The experimental mole ratio of \ce{H2O} to \ce{CuSO4} was \(5.00 \pm 0.002\) with a percent 
    error of 0.033. The RSD value of \(0.41\) is extremely low and indicates a highly precise experiment. 
    The experimental mole ratio of \ce{So4} to \ce{Cu} was \(1.00 \pm 0.0007\) with a percent 
    error of 0.0002. The RSD value of \(0.69\) is extremely low and also indicates a highly precise experiment.
\end{document}