\documentclass{article}
\usepackage{amsmath, amssymb, amsthm}
\usepackage{inputenc}
\usepackage{geometry}
\usepackage{graphicx}
\usepackage{setspace}
\usepackage[version=4]{mhchem}
\geometry{legalpaper, portrait, margin=1in}
\title{Solubility: The Conversion of Mg Metal to Several Magnesium Salts 
\\Experiment 6}
\author{Dylan VanAllen}
\date{Thursday \\ 3/25/2021}
\doublespacing
\begin{document}
    \begin{singlespace}
    \maketitle
    \end{singlespace}
    \section*{Objective}
    In this experiment the solubility of Magnesium salt is explored. Solutions of \(\ce{MgCl2}\), \(\ce{Mg(OH)2}\), \(\ce{MgSO4}\), and \(\ce{MgCO3}\) will be synthesized and a percent yield of \(\ce{MgCO3}\) will be determined for each trial from its theoretical yield.
    \section*{Hypothesis}
    By using our reaction equations below and assuming there will be a percent yield of ~\(90\) percent, the mass of the magnesium carbonate formed  from ~\(0.1\) g \(\ce{Mg^2+}\) should be ~\(0.35 \pm 0.05\) g. Magnesium metal \(\ce{Mg(s))}\) will first react with hydrogen chloride \(\ce{HCL}\) in a single displacement reaction to form magnesium chloride \(\ce{MgCl2(aq)}\) and hydrogen gas \(\ce{H2(g)}\).  Then the magnesium chloride will react with sodium hydroxide \(\ce{NaOH(aq)}\) in a double displacement reaction to form magnesium hydroxide \(\ce{Mg(OH)2(s)}\) and sodium chloride \(\ce{NaCl(aq)}\). Once magnesium hydroxide is made it will react with sulfuric acid \(\ce{H2SO4}\) in another double displacement reaction to form solid \(\ce{MgSO4}\). This magnesium su;fate will then react with sodium carbonate \(\ce{Na2CO3}\) in a double displacement reaction to form solid \(\ce{MgCO3}\). 
    \section*{Variables}
    The independent variable in this experiment is the initial amount of Magnesium metal reacted to form solution. The dependent variable is the yield of magnesium carbonate. 
    \section*{Experimental Outline}
    Use approximately\(0.1\) g of magnesium metal and react it with \(6\) mL of \(3 \ M \ \ce{HCl}\). React the precipitate formed with \(5\) mL of \(5 \ M \ \ce{NaOH}\) and again perform a reaction with the precipitate and \(6\) mL of \(2 \ M \ \ce{H2SO4}\) to get another magnesium salt precipitate. Finally react this product with \(20\) mL of \(1 \ M \ \ce{Na2CO3}\) and record the mass of product formed. Using this mass, determine the percent yield from the theoretical yield. Repeat this twice more and record the average percent yield with its standard deviation and uncertainty. 
    \section*{Key Math Equations}
    \begin{align}
        \ce{Mg(s)}+\ce{2HCl(aq) &-> MgCl2(aq)}+\ce{H2(g)} \\
        \ce{MgCl2(aq)}+\ce{2NaOH(aq) &-> Mg(OH)2(s)}+\ce{2NaCl(aq)} \\
        \ce{Mg(OH)2(aq)}+\ce{H2SO4(aq) &-> MgSO4(s)}+\ce{2H2O(l)} \\
        \ce{MgSO4(aq)}+\ce{Na2CO3(aq) &-> MgCO3(s)+Na2SO4(aq)}
    \end{align}
    \section*{Chemical Hazards and Waste}
    Lab-wear must be worn. Hydrogen gas is involved and is flammable. Magnesium chloride is tosic and magnesium hydroxide is corrosive. Sulfuric acd is also corrosive and magnesium sulfate is toxic. All waste must be disposed of in the proper container. 
    \section*{Conclusion}
    The experiment produced an average mass of \(\hat{m}=0.344\pm0.0037\) g \(\ce{MgCO3}\) which agrees with the hypothesis within uncertainty. The average percent yield of magnesium carbonate was \(93.2\)\% with a standard deviation of \(1.86\) and an RSD of \(19.9\). Since this is in parts per thousand, the precision on this experiment is extremely high. The percent yield being a reasonable value indicates a more accurate experiment. 
\end{document}