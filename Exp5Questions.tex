\documentclass{article}
\usepackage{amsmath, amssymb, amsthm}
\usepackage{inputenc}
\usepackage{geometry}
\usepackage[version=4]{mhchem}
\geometry{legalpaper, portrait, margin=1in}
\begin{document}
ACHM124

Experiment 5 Questions 

Dylan VanAllen 
\bigskip
\par
Complete the following questions for experiment 5.
\begin{enumerate}
    \item Lactic acid is a metabolite formed 
    in the body during muscular activity. It is composed of 
    \(40.00\%\) carbon, \(6.71\%\) hydrogen and \(53.29\%\) oxygen by weight. 
    What is the empirical formula of lactic acid?

    Assume \(100 \ g\):
    
    \begin{align*}
        m_C = 40 \ g = \frac{40 g}{12.011 g/mol} &= 3.33 \ mol \ C \\
        m_H = 6.71 \ g = \frac{6.71}{1.008} &= 6.66 \ mol \ H \\
        m_O = 53.29 \ g = \frac{53.29}{15.999} &= 3.33 \ mol \ O
    \end{align*}
    
    Thus the empirical formula is \(\ce{C3H6O3}\). 
    \item Balance the reaction.
    \begin{equation*}
        \ce{2AgNO3(aq)} + \ce{CaCl2(aq) -> 2AgCl(s)}+\ce{Ca(NO3)2(aq)}
    \end{equation*}
    \item How many grams of iron (II) chloride reacted to form \(15.786 \ g\)
    of aluminum chloride?
    \begin{equation*}
        \ce{2Al(s)} + \ce{3FeCl2(aq) -> 2AlCl3(aq)}+\ce{3Fe(s)}
    \end{equation*}
    
    \(3\) moles of iron (II) chloride is needed for \(2\) moles of aluminum chloride. To get the mass 
    needed we must convert using the molar masses. 
    \begin{equation*}
        15.786 \ g \ \ce{AlCl3} \times \frac{1 \ mol \ \ce{AlCl3}}{133.33 \ g \ \ce{AlCl3}} 
        \times \frac{3 \ mol \ \ce{FeCl2}}{2 \ mol \ \ce{AlCl3}} \times 
        \frac{126.75 \ g \ \ce{FeCl2}}{1 \ mol \ \ce{FeCl2}} = 22.5 \ g \ \ce{FeCl2}
    \end{equation*}
    \item What is the molarity of \(\ce{Na+}\) ions in a \(0.20 \ M \ \ce{Na2CO3}\) solution? 
    \begin{equation*}
        \ce{Na2CO3(s) -> 2Na+(aq)}+\ce{CO3^{2-}(aq)}
    \end{equation*}
    \begin{align*}
        M_{\ce{Na+}} &= M_{\ce{Na2CO3}} \times \frac{mol \ \ce{Na+}}{mol \ ce{Na2CO3}} \\
        M_{\ce{Na+}} &= (0.2) \times \frac{2}{1} = 0.4 \ M
    \end{align*}
    \item How many grams of water are produced if \(27.4 \ mL\) of the \(0.164 \ M\) calcium 
    hydroxide solution is reacted with excess hydrochloric acid?
    \begin{equation*}
        \ce{Ca(OH)2(aq)} + \ce{2HCl(aq) -> CaCl2(aq)}+\ce{H2O(l)}
    \end{equation*}
    To get the amount in moles of \(\ce{Ca(OH)2}\):
    \begin{align*}
        M_{\ce{Ca(OH)2}} &= \frac{mol \ \ce{Ca(OH)2}}{liters \ \ce{Ca(OH)2}} \\
        mol \ \ce{Ca(OH)2} &= M_{\ce{Ca(OH)2}} \times \ liters \ \ce{Ca(OH)2} = 0.164 \times .0274 = .0045 \ mol \ \ce{Ca(OH)2}
    \end{align*}
    The amount of grams of water is now given by:
    \begin{equation*}
        0.0045 \ mol \ \ce{Ca(OH)2} \times \frac{1 \ mol \ \ce{H2O}}{1 \ mol \ \ce{Ca(OH)2}} \times 
        \frac{18 \ g \ce{H2O}}{1 \ mol \ \ce{H2O}} = 0.081 \ g \ \ce{H2O}
    \end{equation*}
    \item When copper (II) chloride reacts with sodium carbonate in an aqueous solution,
    copper (II) carbonate and sodium chloride are produced. 
    \begin{itemize}
        \item Write a balanced equation for the reaction described above.
        \begin{equation*}
            \ce{CuCl2} + \ce{Na2CO3 -> CuCO3}+\ce{2NaCl}
        \end{equation*}
        \item If \(1.530 \ g\) of copper (II) chloride reacts completely with
        \(5.210 \ g\) of sodium carbonate, how much copper (II) carbonate can 
        be formed when the reaction has completed?
        \begin{align*}
            1.53 \ g \ \ce{CuCl2} \times \frac{1 \ mol}{134.45 \ g} &= 0.01138 \ mol  \ce{CuCl2} \\
            5.21 \ g \ \ce{Na2CO3} \times \frac{1 \ mol}{105.99 \ g} &= 0.04916 \ mol  \ce{Na2CO3}
        \end{align*}
        Thus the limiting reactant is \(\ce{CuCl2}\) and the amount of \(\ce{CuCO3}\) formed is found by:
        \begin{equation*}
            0.01138 \ mol \ \ce{CuCl2} \times \frac{1 \ mol \ \ce{CuCO3}}{1 \ mol \ \ce{CuCl2}} \times 
            \frac{123.55 \ g \ \ce{CuCO3}}{1 \ mol \ \ce{CuCO3}} = 1.41 \ g \ \ce{CuCO3}
        \end{equation*} 
    \end{itemize}
\end{enumerate}
\end{document}